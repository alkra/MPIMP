\section{Analyse des performances}

\question
\label{mandel:q:4}

\begin{quotation}
  Pour chaque stratégie de parallélisation :

  \begin{enumerate}
  \item Calculer le temps d'exécution, l'accélération et l'efficacité
    de votre code parallèle en changeant le nombre de processus.
  \item Analyser les résultats obtenus.
  \item Montrer le temps d'exécution de chaque processus (maître,
    esclave) en changeant le nombre de processus.
  \item Analyser les résultats obtenus.
  \end{enumerate}
\end{quotation}

\subsubsection{Équilibrage statique : 2 processus}

Dans cette partie, nous allons étudier l'amélioration apportée par la
parallélsation à deux processus sur la même machine, comparée à
l'algorithme séquentiel.

La machine est un ordinateur pas trop moderne, mais avec un bon
processeur à quatre cœurs et hyperthreading.

\begin{figure}
  \centering

  \begin{tikzpicture}
    \begin{axis}[
      xlabel={Taille de l'image (px$^2$)},
      ylabel={Temps de calcul (s)},
      legend style={
        at={(1,0)},
        anchor=south east
      }
      ]
      \addplot[black, dashed] table [x expr=\thisrowno{1}^2, y index=4,
      col sep=semicolon, header=false]{%
        \data/mandel-seq-taille-ext.csv%
      };

      \addplot[blue] table [x expr=\thisrowno{2}^2, y index=5,
      col sep=semicolon, header=false]{%
        \data/mandel-stat-taille-1-2.csv%
      };

      \legend{séquentiel, statique}
    \end{axis}
  \end{tikzpicture}

  \caption{Comparaison avec l'algorithme séquentiel : profondeur
    constante}
  \label{fig:mandel:stat:comp-seq-taille}
\end{figure}

\begin{figure}
  \centering

  \begin{tikzpicture}
    \begin{axis}[
      xlabel={Profondeur},
      ylabel={Temps de calcul (s)},
      legend style={
        at={(1,0)},
        anchor=south east
      }
      ]
      \addplot[black, dashed] table [x index=3, y index=4,
      col sep=semicolon, header=false]{\data/mandel-seq-prof.csv};

      \addplot[blue] table [x index=4, y index=5,
      col sep=semicolon, header=false]{%
        \data/mandel-stat-prof-1-2.csv%
      };

      \legend{séquentiel, statique}
    \end{axis}
  \end{tikzpicture}

  \caption{Comparaison avec l'algorithme séquentiel : taille
    constante}
  \label{fig:mandel:stat:comp-seq-prof}
\end{figure}

Dans les figures \ref{fig:mandel:stat:comp-seq-taille} et
\ref{fig:mandel:stat:comp-seq-prof}, on remarque que la
parallélisation ne devient intéressante qu'à partir d'un certain
volume de calcul. En l'occurrence, pour une image de côté inférieur
à 2500 pixels à la profondeur 1000, ou une image de côté 1000 à une
profondeur inférieure à 7000, sur cette machine, l'algorithme
parallélisé est plus lent que l'algorithme séquentiel.

La diminution de la pente est due à l'exploitation de deux processeurs
au lieu d'un : les calculs se font plus vite. Pour expliquer la hausse
de l'ordonnée à l'origine, on peut supposer qu'elle est due au temps
de communication entre processus.


\subsubsection{Équilibrage statique : $n$ processus sur une machine}

Dans cette section, sur la nous faisons varier le nombre de processus,
sur une (figure \ref{fig:mandel:stat:one-nproc}) puis six machines
semblables.

\begin{figure}
  \centering

  \begin{tikzpicture}
    \begin{axis}[
      xlabel={Nombre de processus},
      ylabel={Temps de calcul (s)},
      legend style={
        at={(1,0)},
        anchor=south east
      }
      ]
      \addplot[blue] table [x index=1, y index=5,
      col sep=semicolon, header=false]{%
        \data/mandel-stat-proc-1-1000.csv%
      };

      \addplot[red] table [x index=1, y index=5,
      col sep=semicolon, header=false]{%
        \data/mandel-stat-proc-1-5000.csv%
      };

      \legend{1000~px, 5000~px}
    \end{axis}
  \end{tikzpicture}

  \caption{Temps de calcul en fonction du nombre de processus ;
    prondeur constante, images de côté 1000~px ou 5000~px}
  \label{fig:mandel:stat:one-nproc}
\end{figure}

Sur une seule machine, l'évolution du temps de calcul est bien visible
pour les grosses images : quand le nombre de processus est trop faible
(2 ou 4), le temps de calcul augmente ; pour un nombre de processus
deux fois supérieur au nombre de cœurs (ici, 8), on observe une nette
diminution du temps de calcul. Jusqu'à six fois le nombre de cœurs, le
temps de calcul diminue encore, mais moins rapidement ; au-delà, le
calcul est ralenti. On peut supposer que ce ralentissement est dû aux
communications entre les processus, de plus en plus nombreuses.

% Temps de calcul 6 vs 1 machine (image varie)
\begin{figure}
  \centering

  \begin{tikzpicture}
    \begin{axis}[
      xlabel={Taille de l'image (px$^2$)},
      ylabel={Temps de calcul (s)},
      legend style={
        at={(1,0)},
        anchor=south east
      }
      ]
      \addplot[blue] table [x expr=\thisrowno{2}^2, y index=5,
      col sep=semicolon, header=false]{%
        \data/mandel-stat-taille-1-2.csv%
      };

      \addplot[red] table [x expr=\thisrowno{2}^2, y index=5,
      col sep=semicolon, header=false]{%
        \data/mandel-stat-taille-6-2.csv%
      };

      \legend{1 machine, 6 machines}
    \end{axis}
  \end{tikzpicture}

  \caption{Comparaison entre une et six machines : profondeur
    constante (2 processus)}
  \label{fig:mandel:stat:comp-un-six-taille}
\end{figure}

\begin{figure}
  \centering

  \begin{tikzpicture}
    \begin{axis}[
      xlabel={Profondeur},
      ylabel={Temps de calcul (s)},
      legend style={
        at={(1,0)},
        anchor=south east
      }
      ]
      \addplot[blue] table [x index=4, y index=5,
      col sep=semicolon, header=false]{%
        \data/mandel-stat-prof-1-2.csv%
      };

      \addplot[red] table [x index=4, y index=5,
      col sep=semicolon, header=false]{%
        \data/mandel-stat-prof-6-2.csv%
      };

      \legend{1 machine, 6 machines}
    \end{axis}
  \end{tikzpicture}

  \caption{Comparaison entre une et six machines : taille
    constante (2 processus)}
  \label{fig:mandel:stat:comp-un-six-prof}
\end{figure}

Des deux figures \ref{fig:mandel:stat:comp-un-six-taille} et
\ref{fig:mandel:stat:comp-un-six-prof}, on peut remarquer que le temps
de calcul évolue toujours linéairement, mais avec un gain de temps de
près d'une seconde avec six machines. Étant donné qu'on n'utilise que
deux processus, ils ne peuvent tourner que sur deux machines
différentes (et non six). De ces résultats, force est de constater que
l'exécution sur deux machines différentes est bien plus performante
que sur deux cœurs de la même machine. Il doit s'agir d'une
caractéristique purement matérielle. Il peut éventuellement y avoir un
léger aléat dû aux conditions d'exécution : lors des mesures de
performance, j'avais sur la machine principale un environnement de
bureau complet, ainsi que plusieurs fenêtres ouvertes, ce qui peut
occuper les processus. Néanmoins, cet aléat ne peut suffire pour
expliquer la différence de performance, puisque parmi les deux
machines utilisées on retrouve celle sur laquelle je travaillais.

\subsubsection{Équilibrage statique : $n$ processus sur $p$ machines}

% Temps de calcul 6 vs 1 machine (nproc varie)
\begin{figure}
  \centering

  \begin{tikzpicture}
    \begin{axis}[
      xlabel={Nombre de processus},
      ylabel={Temps de calcul (s)},
      legend style={
        at={(1,0)},
        anchor=south east
      }
      ]

      \addplot[blue] table [x index=1, y index=5,
      col sep=semicolon, header=false]{%
        \data/mandel-stat-proc-6-1000.csv%
      };

      \addplot[red] table [x index=1, y index=5,
      col sep=semicolon, header=false]{%
        \data/mandel-stat-proc-6-5000.csv%
      };

      \addplot[blue, dashed] table [x index=1, y index=5,
      col sep=semicolon, header=false]{%
        \data/mandel-stat-proc-1-1000.csv%
      };

      \addplot[red, dashed] table [x index=1, y index=5,
      col sep=semicolon, header=false]{%
        \data/mandel-stat-proc-1-5000.csv%
      };

      \legend{1000~px, 5000~px}
    \end{axis}
  \end{tikzpicture}

  \caption{Temps de calcul en fonction du nombre de processus ;
    prondeur constante, images de côté 1000~px ou 5000~px}
  \label{fig:mandel:stat:six-nproc}
\end{figure}

Dans la figure \ref{fig:mandel:stat:six-nproc}, il faut interpréter
les courbes pleines (six processus) comme étant identiques à la courbe
tiretée (un processus), mais de moindre ordonnée et étirée en
abscisse. Ce graphique signifie que :

\paragraph{Diminution de l'ordonnée}
Quelle que soit la taille de l'image, l'ordonnée est plus faible avec
plusieurs processus : cela marque une diminution du temps de
calcul. Cela rejoint la conclusion de la précédente analyse : un même
nombre de processus travailleront plus rapidement sur plusieurs
machines que sur plusieurs cœurs d'une même machine.

\paragraph{Diminution de la pente}
Pour la courbe où la taille de l'image est $1000$ pixels, on remarque
une diminution de la pente. Cette diminution est aussi présente sur
l'autre courbe (en rouge), même si on ne peut pas réellement parler de
\emph{pente} ; on peut en revanche considérer que les courbes sont
toutes deux étirées selon l'axe des abscisses. On en déduit que le
nombre de processus nécessaire à l'inversion de la tendance (pour la
courbe rouge) est bien plus grand lorsqu'ils sont répartis sur six
machines. Cela s'explique aisément par le fait que davantage de cœurs
de calcul sont disponibles. Pour quantifier ce phénomène, il aurait
fallu réaliser une troisième série de mesures, par exemple avec trois
machines, calculer la pente de chacune, et comparer leur évolution,
par exemple en en comparant la pente ; on aurait alors obtenu une
dérivée seconde, qui correspond à l'accélération du temps de calcul
apportée par plusieurs machines.


\subsubsection{Équilibrage dynamique : avantage sur la méthode
  précédente}


\begin{figure}
  \centering

  \begin{tikzpicture}
    \begin{axis}[
      xlabel={Taille de l'image (px$^2$)},
      ylabel={Temps de calcul (s)},
      legend style={
        at={(1,0)},
        anchor=south east
      }
      ]
      \addplot[black, dashed] table [x expr=\thisrowno{1}^2, y index=4,
      col sep=semicolon, header=false]{%
        \data/mandel-seq-taille-ext.csv%
      };

      \addplot[blue, dashed] table [x expr=\thisrowno{2}^2, y index=5,
      col sep=semicolon, header=false]{%
        \data/mandel-stat-taille-1-2.csv%
      };

      \addplot[blue] table [x expr=\thisrowno{3}^2, y index=6,
      col sep=semicolon, header=false]{%
        \data/mandel-dyn-taille-1-2.csv%
      };

      \legend{séquentiel, statique, dynamique}
    \end{axis}
  \end{tikzpicture}

  \caption{Comparaison avec les algorithmes précédents : profondeur
    constante}
  \label{fig:mandel:dyn:comp-stat-taille}
\end{figure}

\begin{figure}
  \centering

  \begin{tikzpicture}
    \begin{axis}[
      xlabel={Profondeur},
      ylabel={Temps de calcul (s)},
      legend style={
        at={(1,0)},
        anchor=south east
      }
      ]
      \addplot[black, dashed] table [x index=3, y index=4,
      col sep=semicolon, header=false]{\data/mandel-seq-prof.csv};

      \addplot[blue, dashed] table [x index=4, y index=5,
      col sep=semicolon, header=false]{%
        \data/mandel-stat-prof-1-2.csv%
      };

      \addplot[blue] table [x index=5, y index=6,
      col sep=semicolon, header=false]{%
        \data/mandel-dyn-prof-1-2.csv%
      };

      \legend{séquentiel, statique, dynamique}
    \end{axis}
  \end{tikzpicture}

  \caption{Comparaison avec les algorithmes précédents : taille
    constante}
  \label{fig:mandel:dyn:comp-stat-prof}
\end{figure}

\begin{figure}
  \centering

  \begin{tikzpicture}
    \begin{axis}[
      xlabel={Nombre de processus},
      ylabel={Temps de calcul (s)},
      legend style={
        at={(1,1)},
        anchor=north east
      }
      ]
      \addplot[blue, dashed] table [x index=1, y index=5,
      col sep=semicolon, header=false]{%
        \data/mandel-stat-proc-1-1000.csv%
      };

      \addplot[blue] table [x index=1, y index=6,
      col sep=semicolon, header=false]{%
        \data/mandel-dyn-proc-1-1000.csv%
      };

      \addplot[red, dashed] table [x index=1, y index=5,
      col sep=semicolon, header=false]{%
        \data/mandel-stat-proc-1-5000.csv%
      };

      \addplot[red] table [x index=1, y index=6,
      col sep=semicolon, header=false]{%
        \data/mandel-dyn-proc-1-5000.csv%
      };

      \legend{1000~px stat., 1000~px dyn., 5000~px stat., 5000~px dyn.}
    \end{axis}
  \end{tikzpicture}

  \caption{Comparaison avec les algorithmes précédents : nombre de
    processus}
  \label{fig:mandel:dyn:comp-stat-nproc}
\end{figure}

Les figures \ref{fig:mandel:dyn:comp-stat-taille},
\ref{fig:mandel:dyn:comp-stat-prof} et
\ref{fig:mandel:dyn:comp-stat-nproc} concordent : le temps de calcul
en équilibrage dynamique ne varie presque pas par rapport à
l'équilibrage statique. Cela nous rappelle que toutes les zones de
l'image prennent rigoureusement autant de temps à calculer (voir le
paragraphe correspondant à la question \ref{mandel:q:2}), donc un
équilibrage dynamique n'était pas particulièrement requis. On déduit
également de ces résultats qu'il ne faut pas surestimer l'importance
des communications : en effet, bien que l'algorithme dynamique
requière beaucoup plus de communications, le temps de calcul
n'augmente pas. Cette dernière remarque est néanmoins à tempérer,
puisque les mesures ont été faites sur la \emph{même machine} ; nous
n'avons hélas !\footnote{Cela prend un temps fou !} pas de mesures
semblables réparties sur plusieurs machines.



\subsubsection{Équilibrage dynamique : évolution de la taille de blocs}


\begin{figure}
  \centering

  \begin{tikzpicture}
    \begin{axis}[
      xlabel={Nombre de blocs},
      ylabel={Temps de calcul (s)},
      legend style={
        at={(1,0.5)},
        anchor=east
      }
      ]
      \addplot[blue] table [x index=2, y index=6,
      col sep=semicolon, header=false]{%
        \data/mandel-dyn-nblocks-1-1000.csv%
      };

      \addplot[red] table [x index=2, y index=6,
      col sep=semicolon, header=false]{%
        \data/mandel-dyn-nblocks-1-5000.csv%
      };

      \legend{1000~px, 5000~px}
    \end{axis}
  \end{tikzpicture}

  \caption{Évolution du temps de calcul en fonction du nombre de
    blocs}
  \label{fig:mandel:dyn:nblocks}
\end{figure}

La figure \ref{fig:mandel:dyn:nblocks} mesure le temps de calcul en
fonction du nombre de blocs sur deux images de taille différente. On
rappelle que plus on diminue la taille des blocs, plus leur nombre
augmente. Pour les deux courbes, plus le nombre de blocs augmente
(\emph{ie.} leur taille diminue), plus la pente augmente. Cela
traduit un ralentissement de la vitesse de calcul, ce qui peut
s'expliquer par le plus grand délai, dû aux communications plus
fréquentes entre le processus maître et les processus enfants.


%%% Local Variables:
%%% mode: latex
%%% TeX-master: "../rapport"
%%% End:
