\chapter{Makefiles}
\label{app:Makefiles}


\section{Introduction}

Au fur et à mesure que les projets grossissent, le temps passé à
compiler les codes sources augmente. GNU \texttt{make} est un
programme permettant de ne recompiler que les fichiers qui ont changé
depuis la précédente compilation. Il faut néanmoins le configurer
proprement, dans un fichier nommé \texttt{Makefile}.

Un Makefile se présente sous la forme d'une liste de cibles
(\emph{targets}), chacune suivie d'éventuelles cibles dont elle dépend
et d'une liste d'instructions shell. Exemple ci-dessous :

\begin{verbatim}
cible: dep1 dep2
        instruction arg1 arg2
        autre > fichier
\end{verbatim}

Une cible peut être un fichier, auquel cas elle ne sera réévaluée que
s'il a été changé, ou une \emph{PHONY target}, c'est-à-dire un simple
nom facile à retenir pour l'utilisateur final. Pour ne pas que
\texttt{make} confonde cette dernière avec un fichier, il est d'usage
de la mettre en dépendance d'une cible au nom spécial :
\texttt{.PHONY}. Pour l'exemple, nous allons commenter le Makefile du
TP sur la fractale en version MPI (listing \ref{lst:mandel:Makefile}).

\lstinputlisting[float=*, language={[gnu]make},
  caption=Makefile de Mandelbrot en version MPI,
  label=lst:mandel:Makefile,
  linerange={10-12, 14-29, 32-39}
]{\fragment/mpi-mandel-Makefile}

Après les commentaires d'en-tête non repris ici, on commence par une
première cible. C'est celle qui sera exécutée lorsque le programme
\texttt{make} est appelé sans cible en argument. Cette cible renvoie,
par un jeu de dépendance, vers une cible définie plus bas. Notons que
cette première cible n'est pas un nom de fichier ; elle apparaît donc
dans la liste des dépendances de \texttt{.PHONY} tout en bas du
fichier.

Ensuite apparaissent deux cibles représentant les fichiers objets
intermédiaires. Ils peuvent être compilés et liés en un seul appel au
compilateur, mais le but de \texttt{make} étant d'éviter de tout
recompiler à chaque fois, il est recommandé de les séparer en
plusieurs cibles. Évidemment, cela n'a que peu d'importance pour ce
projet à un seul fichier source. Les deux cibles sont en réalité deux
versions du même fichier source : la première est une version de
débogage, où l'on demande au compilateur de générer des avertissements
et des symboles de débogage, et la deuxième une version de production,
où le compilateur optimise un peu le code.

Conformément aux deux versions mentionnées plus haut, l'appel à
l'éditeur de liens se fait dans deux cibles distinctes : une cible de
production, \texttt{release}, réclamant le fichier \texttt{mandel.o}
issu de la compilation en mode production ; et une cible de débogage,
\texttt{debug}, réclamant le fichier \texttt{mandel.od} issu de la
compilation de débogage. Ces deux cibles sont des noms qui parlent à
l'utilisateur et non pas des noms de fichier\footnote{En
  l'occurrence, elles renvoient toutes deux au même fichier,
  \texttt{./mandel.akraus}} ; elles figurent donc dans les dépendances
de la fausse-cible \texttt{.PHONY} en bas du fichier.

Enfin, nous proposons deux dernières ci\-bles dans ce Makefile
permettant respectivement d'enlever tous les fichiers qui ne sont pas
présents dans le répertoire de base, et de relancer une compilation en
partant de zéro (notez l'utilisation des dépendances pour cette
dernière).


\section[Makefile du rapport]{%
  Un example avancé : le Makefile du rapport%
}

Ce Makefile, bien que plus long, n'est pas beaucoup plus technique que
le précédent.

Il commence par la définition d'un certain nombre de variables. Les
premières sont les programmes à utiliser, qui peuvent être dépendants
de l'installation (par exemple, préciser le chemin complet). Ils
peuvent être aisément modifiés, tant qu'ils présentent la même
interface. Ensuite vient la liste des fichiers source, qui n'est pas
utilisée, et la liste des fichiers intermédiaires nécessaires à la
compilation du rapport, qui est utilisée inaltérée à plusieurs
endroits.

Les fichiers intermédiaires sont issus des gros fichiers CSV de
mesures, et sont destinés à être importés directement dans le rapport
: ils sont donc découpés par série de mesures. Commençons par regarder
la cible d'un tel fichier intermédiaire en listing
\ref{lst:mandel:stat} page \pageref{lst:mandel:stat}.

\lstinputlisting[float=*, language={[gnu]make},
  caption=Makefile du rapport (extrait : CSV intermédiaire),
  label=lst:mandel:stat,
  linerange={60-67, 72-75, 89-92}
]{\fragment/rapport-Makefile}

\lstinputlisting[float=*, language={TeX},
  caption=Utilisation de l'extrait de la figure \ref{lst:mandel:stat}
  dans le fichier \LaTeX,
  label=lst:rapport:stat,
  firstline=45, lastline=48
]{mandelbrot/question-3.tex}

La cible du bas est le nom du fichier à générer. \texttt{make}
commence par exécuter la dépendance :\\
\texttt{../Mandelbrot/mandel-stat.csv}.

Cette cible de dépendance se trouve cachée dans le groupe intitulé :\\
\texttt{../Resultats/\%-stat.csv}, où \texttt{\%} est un caractère
\emph{joker}. Le nom complet de la cible appelée se trouve alors dans
la variable \texttt{\$@}. Cet astuce nous permet de traiter les
résultats de Convolution et de Mandelbrot de la même manière. Cette
cible n'a aucune dépendance, donc les instructions sont ex\-écu\-tées
directement. On récupère le fichier ciblé sur sa branche
\emph{mpi-static}. Par défaut, Git le porte candidat comme fichier à
valider ; on le repasse alors au statut de fichier non suivi avec
\texttt{git reset}.

Puis, \texttt{make} revient sur la cible première. Le fichier
\texttt{mandel-stat.csv} contient l'intégralité des mesures ; or, pour
le rapport, on ne veut que les mesures avec deux processus sur la même
machine, avec la taille de l'image à générer variant (d'où le nom du
fichier). On va donc sélectionner les lignes du fichier
correspondantes avec une instruction \texttt{sed}, où le programme
\texttt{sed} est caché dans la variable \emph{line}.

Utiliser un processus aussi compliqué a presque plus d'inconvénients
que d'avantages. Si on applique un traitement mineur sur les mesures,
par exemple convertir les secondes en millisecondes, le rapport sera
compilé avec les nouvelles valeurs de manière transparente ; il
suffira de changer la légende des graphiques. Si on intervertit deux
colonnes, la compilation fonctionnera ; il faudra néanmoins penser à
modifier les codes \LaTeX\ correspondants (exemple en listing
\ref{lst:rapport:stat}). Si on déplace des lignes, il n'y aura pas
besoin de modifier le code \LaTeX, mais en revanche il faudra
retravailler le Makefile. En somme, une telle démarche ne simplifie
pas la gestion des données externes, et n'est donc valable qu'à titre
d'exercice.

En pratique, on n'appellera presque jamais la cible d'un fichier
intermédiaire. Lors du développement du rapport, je disposais d'un
éditeur de texte qui se chargeait d'appeler l'interpréteur \LaTeX ; je
me contentais alors d'exécuter \texttt{make data} (et \texttt{make
  clean} avant de changer de branche). La cible \texttt{data} dépend
de tous les petits fichiers intermédiaires, qui seront générés l'un
après l'autre. Enfin, on applique un traitement sur tous ces petits
fichiers pour convertir la virgule rentrée par LibreOffice lors de
l'édition des fichiers CSV originaux en point décimal ; le standard
CSV utilisé par l'interpréteur côté \LaTeX\ utilise en effet le point
comme séparateur décimal. Détaillons l'expression régulière utilisée
pour cette tâche :

\begin{description}
\item[s///g :] remplacer (\emph{\textbf{s}ubstitute}) toutes les
  occurrences dans la ligne considérée (\emph{\textbf{g}lobally}) de
  ce qui est entre les deux premiers / par ce qui est entre les deux
  derniers / ;
\item[{[0-9]}] un chiffre ;
\item[\textbackslash ( \textbackslash )] ce qui est
  là-dedans constitue un des neuf groupes possibles, à mettre dans
  \textbackslash 1, puis \textbackslash 2, etc. ;
\item[{[0-9],[0-9]*}] un chiffre suivi
  d'une virgule, suivie de 0 ou plusieurs chiffres ; à remplacer par :
\item[\textbackslash 1\textbackslash .\textbackslash 2] le premier
  groupe (= l'unité), suivi d'un point (échappé car c'est un caractère
  réservé), suivi du deuxième groupe (= la partie décimale)
\end{description}

Évidemment, cette expression régulière est un peut trop simple et peut
s'appliquer à autre-chose que des nombres : par exemple, \og Le chef
compta 1,2,3 et les musiciens partirent.\fg\ Cependant, les données
que nous avons ici sont convenables.

Dans le code \LaTeX\ du listing \ref{lst:rapport:stat}, on calcule à
la volée le carré de la taille de l'image (deuxième colonne, numérotée
à partir de 0) en abscisse, et l'ordonnée est la mesure dans la
cinquième colonne. Les autres options précisent que le séparateur du
fichier texte est un point-virgule, et que la série commence à la
première ligne du fichier.



%%% Local Variables:
%%% mode: latex
%%% TeX-master: "../rapport"
%%% End:
