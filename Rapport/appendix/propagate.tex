\chapter{Le script propagate}
\label{app:propagate}


Pour décrire ce script, nous allons nous efforcer d'adopter une
progression linéaire.



\section{L'interface}

L'interface se fait sous une forme bien connu des utilisateurs de
ligne de commande sous Linux ; elle est identique pour tous les
scripts propagate de toutes les branches.

\begin{description}
\item[nom du programme] ./propagate ;
\item[option courte] un tiret, une lettre, un espace, la valeur de
  l'option ;
\item[option longue] deux tirets, le nom complet de l'option, un
  espace, la valeur de l'option ;
\item[argument \no 1 (optionnel)] le nombre de processus ;
\item[autres arguments (optionnels)]\null$\;$ des arguments qui
  vont être directement passés à la ligne de commande \texttt{mpirun}.
\end{description}

La boucle de lecture des arguments tour\-ne tant qu'il en reste, mais
sera arrêtée (\texttt{break}) une fois le nombre de processus ou, à
défaut, \texttt{-\null-} lu.

Le premier cas est l'option d'aide ; dans ce cas, le texte d'aide est
affiché et le programme s'arrête. L'affichage du texte d'aide se fait
avec des \texttt{echo}, ce qui est fort disgrâcieux : entre-temps,
j'ai appris à écrire un pseudo-fichier directement dans l'entrée
standard et à l'afficher avec \texttt{cat} :

\begin{lstlisting}[language=bash]
  cat <<EOF
Ceci est le contenu d'un
pseudo-fichier, avec des
retours a la ligne, et
interprete par ${SHELL}.
EOF
\end{lstlisting} % $

Chaque option, ainsi que le premier argument est associé à une
variable. La valeur par défaut est écrite dès le début du programme,
ou laissée vide et écrite après la grande boucle si elle dépend de la
valeur d'une autre option. (par exemple, \texttt{path} dépend de
\texttt{user}). Si l'option est dans la ligne de commande : dans le
premier cas, elle écrase simplement la valeur ; dans le deuxième cas,
elle remplit la variable, qui sera testée après la boucle, pour savoir
si elle est vide, et donc susceptible d'être réinitialisée.


\section{Les fonctions utiles}

Au cours de ce script, il existe des fragments de code souvent
réutilisés, par exemple pour exécuter une commande sur la machine
distante, ou copier un fichier. Nous avons donc écrit des fonctions
réalisant ces opérations.

\begin{description}
\item[runlocalcmd] affiche puis exécute une commande sur la machine
  courante ; la valeur de retour est testée pour savoir si le
  processus entier doit être arrêté ;
\item[runcmd] exécute une commande sur la machine distante : cela
  revient à exécuter en local un client ssh avec la commande
  \texttt{\$*} en paramètre ;
\item[sendfile] envoie un fichier sur la machine distante ; à nouveau,
  cela revient à exécuter \texttt{scp} (pour \emph{\textbf{s}ecure
    \textbf{c}o\textbf{p}y}) en local avec en premier argument le
  paramètre de la fonction ;
\item[getfile] effectue la copie inverse.
\end{description}

La fonction \textbf{sendsourcefile} fait l'objet de la section
suivante.


\section{Gestion des codes sources}

Cette section décrit l'objet d'un exercice de style. Nous souhaitons
ne pas avoir à copier un fichier s'il n'a pas été modifié. En somme,
il s'agit d'imiter le comportement d'un Makefile.

Il est difficile de répondre à ce problème avec un (ou plus difficile,
plusieurs) Makefile à cause du grand nombre de comportements dépendant
d'options en ligne de commande.

Pour savoir si le fichier a été modifié entre deux lancements, nous
allons enregistrer une somme de contrôle dans un fichier de même nom
se trouvant dans le sous-dossier caché \texttt{.sha}. Une somme de
contrôle est un nombre dont le calcul fait intervenir tous les octets
du fichier en entrée : s'il est modifié à un endroit, le nombre
change. L'algorithme utilisé est bien évidemment plus complexe qu'une
simple somme, pour éviter les modifications qui se compensent (exemple
: $4 = 1 + 3 = 2 + 2$). Étant donné que cette application n'a pas de
but cryptographique, l'algorithme utilisé importe peu. Pour calculer
cette somme de contrôle, nous utilisons le programme \texttt{shasum}
(du paquet GNU coreutils).

À présent, nous pouvons commenter la fonction
\texttt{sendsourcefile}. On y commence par vérifier que le fichier à
envoyer existe, puis que le dossier \texttt{.sha} existe (ce qui
devrait toujours être le cas, \emph{cf. infra}). S'il existe, on
compare la somme de contrôle du fichier en argument à celle écrite
dans le fichier de même nom dans le dossier \texttt{.sha}. Si les
sommes diffèrent, ou que le fichier-somme est vide, ou que le dossier
\texttt{.sha} n'existe pas, on envoie le fichier, on en calcule la
somme, qu'on écrit dans un fichier de même nom dans le dossier
\texttt{.sha}. La variable \texttt{COMPILE} indique s'il est
nécessaire de lancer une compilation.

Après la définition de cette fonction commence le code ayant pour but
d'envoyer les fichiers sources. La première étape est de créer le
dossier \texttt{.sha} s'il n'existe pas. On initialise également la
variable \texttt{COMPILE}. Puis, on boucle sur les fichiers sources et
on les envoie si besoin.

Si la variable \texttt{COMPILE} n'est pas vide\footnote{\emph{i.e.}
  elle est à \texttt{true}, mais il est plus robuste de tester si elle
  a été modifiée depuis son initialisation à vide.}, on lance la
compilation et on recopie les exécutables sur toutes les machines du
cluster.

On applique aux fichiers de données\footnote{Le \texttt{in} de la
  boucle est vide, car il n'y a pas de fichier de données pour
  Mandelbrot.} un traitement similaire, en omettant la somme de
contrôle puisqu'ils ne sont pas censés être modifiés. On écrit la
liste des fichiers envoyés (un nom par ligne) dans le fichier caché
\texttt{.sent\_data}. On ne renvoie pas un fichier qui a déjà été
envoyé (et qui est par conséquent dans \texttt{.sent\_data}, ce que le
programme GNU \texttt{grep} permet de vérifier).


\section{Lancement du calcul}

Si toutes les étapes précédentes ont réussi, on peut lancer le calcul
et récupérer le résultat.

\clearpage

\begin{onecolumn}
  \section{Le script complet (Mandelbrot)}

  \lstinputlisting[language=bash,
  basicstyle=\scriptsize, showspaces=false,
  showstringspaces=false,
  linerange={1-18, 58-220},
  caption=Propagate
]{\fragment/propagate}
\end{onecolumn}

\clearpage
\twocolumn



%%% Local Variables:
%%% mode: latex
%%% TeX-master: "../rapport"
%%% End:
