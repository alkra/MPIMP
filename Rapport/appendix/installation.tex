\chapter{Guide d'installation}
\label{app:install}

Le code du TP est versionné à l'aide du système Git. Il est réparti
sur plusieurs branches, qui ne sont pas destinées à être fusionnées
(voir l'annexe \ref{app:git}). Ainsi, la récupération de toutes les
branches est indispensable.

\section{Prérequis}

\subsubsection{Pour le code}

\paragraph{Linux}
Les programmes sont écrits en C, et conçus pour fonctionner sur un
système \emph{à la Unix} (GNU/Linux, BSD, Mac, ...). Windows n'est
officiellement pas supporté ; toutefois, si vous avez les
connaissances requises (adapter les chemins de fichier, installer
les programmes \texttt{GNU}, \texttt{sed}, \texttt{bash}, déboguer les
\texttt{Makefile}s), le projet devrait fonctionner correctement.

\paragraph{GCC}
Il vous faut un compilateur C. Nous utilisons \texttt{gcc}. Si vous
souhaitez utiliser un autre compilateur, il vous faudra adapter les
\texttt{Makefile}s. (voir l'annexe \ref{app:Makefiles}).

\paragraph{OpenMPI}
OpenMPI comprend une bibliothèque C, à installer dans le répertoire
correspondant de votre compilateur, et une collection de binaires
(notamment le pseudo-compilateur \texttt{mpicc} et
\texttt{mpirun}). Les distributions GNU/Linux de type Debian les
distribuent dans les paquets \texttt{libopenmpi-dev} et
\texttt{openmpi-bin}.

\paragraph{OpenMP}
Assurez-vous d'utiliser un com\-pi\-la\-teur compatible avec les
directives OpenMP. Il n'y a rien à installer.

\paragraph{Make}
GNU \texttt{make} est un programme permettant d'automatiser la
compilation (voir annexe \ref{app:Makefiles}). Le paquet Debian a pour
nom \texttt{make}.

\paragraph{Git}
Git est un système de gestion de versions distribué. Pour plus de
détails quant à son utilité pour le projet, reportez-vous à l'annexe
\ref{app:git}.

\paragraph{openssh}
\texttt{openssh} permet d'ouvrir un shell sécurisé sur une machine
distante. Nous en avons besoin pour lancer des programmes MPI sur
plusieurs machines. (voir la section \ref{sec:installation:machines})

\paragraph{bash}
Nos scripts d'automatisation sont à exécuter dans le
\emph{\textbf{B}ourne-\textbf{a}gain \textbf{sh}ell}. \texttt{bash}
devrait être installé sur la plupart des distributions GNU/Linux et
Mac.

\subsubsection{Pour le rapport}

Le rapport est rédigé en \LaTeX. Si vous souhaitez le recompiler, vous
aurez besoin des paquets cités ci-après.


\paragraph{GNU/Linux, Make, Git} \emph{confere supra}.

\paragraph{pdflatex}
Une distribution \TeX\,/\,\LaTeX\ est composée d'un interpréteur
(respectivement \texttt{[pdf]tex} et \texttt{[pdf]latex}) générant un
fichier au format \emph{\textbf{d}e\textbf{v}ice
  \textbf{i}ndependant}, destiné à être compris par toutes les
imprimantes, ou directement dans le \emph{\textbf{p}ortable
  \textbf{d}ocument \textbf{f}ormat} (PDF) qui est mieux adapté à une
lecture dématérialisée ; et d'un grand nombre de paquets qui
enrichissent le langage. L'ensemble est distribué sous le nom de \TeX
live sous GNU/Linux, Mac\TeX\ sous Mac, et Mik\TeX\ sous Windows. Si
vous êtes sous Debian, vous aurez besoin des paquets suivants :
\texttt{texlive-latex-extra}, \texttt{texlive-science}, et
\texttt{texlive-lang-french} ; par le jeu des dépendances, vous
installerez tous les autres paquets nécessaires. Les autres
distributions proposent un moyen d'installer les paquets \LaTeX\ à la
volée.

\section{MPI – Configuration des machines}
\label{sec:installation:machines}

Cette section décrit le script \texttt{init-mpi}.

Le fichier \texttt{hosts} dans les dossiers des sources contient la
liste des machines à utiliser pour lancer les programmes. Il doit y
avoir une adresse IP ou nom de domaine par ligne. Le programme sera
recompilé sur la première machine listée.

Les communications au sein d'un communicateur utilisent des
connections SSH. \emph{Il ne doit pas être demandé de mot de
  passe}. Pour cela, vous avez la possibilité de modifier la
configuration de PAM pour ne pas authentifier la connexion, ou bien
utiliser une authentification à base de clés openssh.

Pour générer une clé, on ouvre un shell sur la première machine, et on
lance la commande \texttt{ssh-keygen -t rsa -b 1024}. Pour la suite de
cette section, il sera considéré que vous avez laissé le nom de
fichier par défaut. Ne pas entrer de mot de passe, tapez \emph{Entrée}
directement aux deux invites.

On génère une clé courte (1024 bits) pour ne pas trop perdre de
temps. Si le système est critique, ou l'utilisation en est prolongée,
augmenter cette valeur.

Le fichier \texttt{id\_rsa} contient la clé par défaut associée à
l'utilisateur.

On ne protège pas la clé par un mot de passe, puisque l'objet de cette
démarche est justement de ne pas avoir à taper de mot de passe.

On va ensuite faire connaître cette clé à toutes nos machines, en
utilisant une boucle \texttt{for} qui épluche le fichier
\texttt{hosts}. Vous êtes encouragé à utiliser le même nom
d'utilisateur sur toutes les machines ; dans le cas contraire, de
nombreux scripts et configuration de ces annexes ne marcheront pas.

\begin{lstlisting}[language=bash]
for host in $(cat hosts)
do
    ssh-copy-id $host
done
\end{lstlisting}

Et on recommance pour toutes les machines. En fait, on peut encapsuler
ces trois commandes dans une boucle \texttt{for} depuis le répertoire
du ficher hosts. Pensez à adapter le dossier de destination.


\section{Récupération des sources}

Les sources sont hébergées en ligne, sur mon GitHub, à l'adresse
\url{https://github.com/alkra/MPIMP}. Pour les télécharger, nous
allons utiliser le client \texttt{git}. Le contenu du dépôt sera copié
dans un sous-dossier MPIMP du dossier courant.

\begin{verbatim}
git clone \
    https://github.com/alkra/MPIMP
\end{verbatim}

Ce n'est pas fini ! Pour l'instant, vous n'avez récupéré que la
branche \emph{master} du dépôt. Vous devez suivre les autres
branches (listing \ref{lst:installation:clone}).

\begin{lstlisting}[float=*, language=bash, caption=Copie du dépôt,
  label=lst:installation:clone]
cd MPIMP
for branch in mpi-master mpi-static mpi-dynamic \
              mp-seq mp-static mp-dynamic
do
    git checkout -b $branch origin/$branch
done
git checkout master
\end{lstlisting}

Voir l'annexe \ref{app:git} pour plus de détails.



\section{Compilation et lancement}

\subsubsection{Programmes séquentiels}

Il suffit de copier (avec \texttt{scp}) le dossier source sur la
machine distante, s'y déplacer, et lancer \texttt{make} suivi de
\texttt{./mandel.akraus}. L'extension de ce fichier exécutable vient
du fait que, dans les conditions du TP, tous les étudiants
travaillaient sur les mêmes machines dans le même dossier : il fallait
donc différentier les exécutables de chacun.

\subsubsection{MPI}

Sur les branches dérivant de \emph{mpi-master} se trouve un script
\texttt{propagate}, décrit en annexe \ref{app:propagate}, qui se
charge de mener à bien la compilation, le lancement, et la
récupération des résultats.

\begin{verbatim}
./propagate --help
./propagate <options>
\end{verbatim}



\section{Nettoyage – todo-list}

Il faut supprimer les fichiers copiés sur toutes les machines
distantes. Vous pouvez faire une boucle \texttt{for} sur le contenu du
fichier hosts.

Sur la première machine, faire un \texttt{make clean} pour supprimer
les exécutables et fichiers de compilation. Supprimez aussi les
fichiers de résultat, ainsi que l'ensemble des sources.

Enlevez les $n$ dernières lignes des fichiers
\texttt{\string~/.ssh/authorized\_keys} de toutes les machines, où $n$
est le nombre de lignes du fichier hosts. Chaque ligne de ceux-là
représente la clé d'un des hôtes. Enfin, sur chaque machine, supprimez
\texttt{\string~/.ssh/id\_rsa} et \texttt{\string~/.ssh/id\_rsa.pub}.




\section{Compiler le rapport}

D'abord, assurez-vous avec un \texttt{git status} que vous n'avez
aucune modification détectée. Si des noms de fichier rouges
apparaissent avant la section \emph{Fichiers non suivis (Untracked
  files)}, validez-les, ou annulez-les avec \texttt{git checkout --
  <fichier>}.

Déplacez-vous ensuite dans le sous-dossier \texttt{Rapport} de la
branche \emph{master}, et lancez \texttt{make}. Cela va générer un
grand nombre de fichiers csv dans \texttt{data}, que vous pourrez
supprimer avec un \texttt{make clean}, afficher beaucoup de lignes sur
votre terminal, et enfin générer un fichier \texttt{rapport.pdf}.

%%% Local Variables:
%%% mode: latex
%%% TeX-master: "../rapport"
%%% End:
