\newenvironment{git}{
  \bigskip
  \noindent%
  \begin{tikzpicture}[scale=0.9]%
}{%
  \end{tikzpicture}
  \bigskip
}

\gdef\ccommit[#1](#2,#3)#4{
  (#2, #3) coordinate (#4)
  circle[fill, radius=0.1]
  node[outer sep=1pt, #1]{#4}
}
\gdef\commit(#1,#2)#3{
  (#1, #2) coordinate (#3)
  circle[fill, radius=0.1]
  node[outer sep=1pt, above]{#3}
}

\newcommand{\bmaster}{
  \commit(0, 0){A} node[outer sep=3pt, below]{v0.0} --
  \commit(1, 0){B} node[outer sep=3pt, below]{v0.1} --
  \commit(5, 0){F} --
  \commit(6, 0){I} node[outer sep=3pt, below]{v1.0}
                   node[outer sep=3pt, right]{\emph{master}}
}

\newcommand{\balpha}{
  (B) --
  \commit(2, -1){C} --
  \commit(3, -1){D} --
  \commit(4, -1){E} node[outer sep=3pt, right]{\emph{alpha}} --
  (F)
}

\newcommand{\bbeta}{
  (B) --
  \commit(2.5, -2){G} --
  \commit(4, -2){H} node[outer sep=3pt, right]{\emph{beta}} --
  (I)
}

\newcommand{\bbetaPrime}{
  (D) --
  \commit(4, -2){G'} --
  \commit(5, -2){H'} node[outer sep=3pt, right]{\emph{beta}} --
  (I)
}

\chapter{Arborescence \texttt{git}esque}
\label{app:git}

\section{À propos de Git}

Git est un système de gestion de version. Il permet d'enregistrer
l'état d'un dossier et de ses sous-dossiers, l'ensemble étant appelé
\emph{dépôt}, à l'époque courante. Un tel enregistrement est appelé
\emph{commit}. À partir de ces enregistrements, il est capable
d'intégrer les changements effectués par un collègue qui utilisait une
ancienne version, il permet de revenir en arrière, de voir
l'historique des modifications...

Un flot de travail couramment utilisé s'appuie sur plusieurs
branches. On donne le nom de \emph{master} au fil des modifications
faites par l'administrateur ; ces modifications correspondent à
l'intégration des fonctionnalités développées par les collaborateurs.

La figure suivante représente ce fil des modifications, où chaque
point correspond à un commit numéroté par une lettre.

\begin{git}
  \draw[thick] \bmaster;
\end{git}

À partir de cette branche principale, un développeur doit implémenter
une fonctionnalié $\alpha$ pour la version 1.0. Pour ce faire, il crée
une nouvelle branche \emph{alpha}, sur laquelle il réalise trois
commits. Il va ensuite soumettre son travail à l'administrateur, qui
va vérifier que l'ensemble fonctionne toujours aussi bien (les tests
passent), que le style de codage est propre, et va ensuite fusionner
cette branche avec la branche principale. L'historique ressemble alors
à ceci :

\begin{git}
  \draw[thick] \bmaster;
  \draw[thick, red] \balpha;
\end{git}

En parallèle, un deuxième développeur implémente la fonctionnalité
$\beta$. De même, l'administrateur, aidé du puissant al\-go\-ri\-thme
de théorie des graphes de Git, va fusionner les modifications
correspondantes dans la branche \emph{master}.

\begin{git}
  \draw[thick] \bmaster;
  \draw[thick, red] \balpha;
  \draw[thick, blue] \bbeta;
\end{git}

Une autre possibilité qu'offre git, que j'utilise abondamment pour ce
projet, est la faculté d'intégrer des changements avant son
travail. Par exemple, supposons que le deuxième développeur se rende
compte qu'il a besoin du commit D pour implémenter sa
fonctionnalité. Il va alors se \emph{rebaser} sur \emph{alpha}-D. Le
graphe devient alors :

\begin{git}
  \draw[thick] \bmaster;
  \draw[thick, red] \balpha;
  \draw[thick, blue] \bbetaPrime;
\end{git}

Il faut noter qu'un des changements introduits par G a déjà pu être
introduit par C ou D ; c'est pourquoi, lors d'un \emph{rebase}, les
commits sont systématiquement renommés, et engendrent tous les
conflits entre utilisateurs liés à la modification de l'historique.



\section{Mon utilisation}

Je suis tout seul sur ce projet. Je n'ai donc pas à me soucier
d'éventuels collaborateurs, et suis libre de modifier l'historique à
ma guise. J'utilise donc allègrement les \emph{rebase}, ce qui me
permet d'avoir un dépôt dont l'historique est suffisamment simple pour
être compréhensible par un humain, et très bien organisé.

En effet, pour ce projet, on réalise diverses modifications, telles
que la modification de l'algorithme pour réaliser un équilibrage de
charges dynamique, l'ajout de directives OpenMP par-dessus,
... Toutefois, certaines sont communes à plusieurs codes : par
exemple, l'ajout de l'appel à \texttt{MPI\_Init} est commun à tous les
codes MPI et MPI+MP.

J'ai donc organisé les différents codes en un arbre, en figure
\ref{fig:git:hierarchie}.

\begin{figure*}
  \centering

  \begin{tikzpicture}
    % MASTER
    \draw[ultra thick]
    \commit(0,0){git} --
    \commit(2,0){codeProf} --
    \commit(4,0){resultSeq};
    \draw[thick, gray]
    (resultSeq) --
    \commit(6,0){rapportBase} --
    \commit(8,0){rapport};
    \draw[thin, gray] (8,0) -- (12,0) node[right]{\emph{master}};

    % MP-SEQ
    \draw[ultra thick, blue]
    (codeProf) --
    \commit(4,-1){MPseq} --
    \commit(6,-1){resultMPseq};
    \draw[thin, blue] (6,-1) -- (12,-1) node[right]{\emph{mp-seq}};

    % MPI-MASTER
    \draw[ultra thick, yellow]
    (codeProf) --
    \ccommit[below left](4, -6){MPIbase}
    node[outer sep=3pt, below right, black]{
      \parbox{10cm}{
        \begin{description}
        \item[Makefile] avec \texttt{mpicc} ;
        \item[hosts] ;
        \item[propagate] ;
        \item[mandel.c, convol.c] code de base MPI.
        \end{description}
      }
    };
    \draw[thin, yellow] (4,-6) -- (12,-6) node[right]{\emph{mpi-master}};

    % MPI-STATIC
    \draw[ultra thick, green]
    (MPIbase) --
    \commit(6, -2){MPIstat} --
    \commit(8.5, -2){resultMPIstat};
    \draw[thin, green] (8.5,-2) -- (12,-2) node[right]{\emph{mpi-static}};

    % MPI-MP-STATIC
    \draw[ultra thick, cyan]
    (MPIstat) --
    \commit(8, -3){MPstat} --
    \commit(10,-3){resultMPstat};
    \draw[thin, cyan] (10,-3) -- (12,-3) node[right]{\emph{mp-static}};

    % MPI-DYNAMIC
    \draw[ultra thick, red]
    (MPIbase) --
    \commit(6, -4){MPIdyn} --
    \commit(8.5,-4){resultMPIdyn};
    \draw[thin, red] (8.5,-4) -- (12,-4) node[right]{\emph{mpi-dynamic}};

    % MPI-MP-DYNAMIC
    \draw[ultra thick, magenta]
    (MPIdyn) --
    \commit(8, -5){MPdyn} --
    \commit(10,-5){resultMPdyn};
    \draw[thin, magenta] (10,-5) -- (12,-5) node[right]{\emph{mp-dynamic}};

  \end{tikzpicture}

  \paragraph{Description des commits :}
  \begin{tabularx}{\textwidth}{r X}
    git & Initialisation d'un dépôt vide et des fichiers propres à
          Git\\
    codeProf & Le code donnée par le prof, modulo quelques
               modifications (encodage et noms de fichiers)\\
    MPIbase & Code de base pour OpenMPI : initialisation, récupération
              du rang, validation des arguments du programme, etc. +
              fichiers mentionnés\\
    MPIstat & Adaptation des codes du commit précédent pour un
              équilibrage des charges statique\\
    MPIdyn  & Idem, mais avec un équilibrage dynamique (uniquement
              \texttt{mandel.c})\\
    MP* & Code du commit * adapté pour OpenMP (parfois seulement
           \texttt{mandel.c})\\
    result* & Mesures de performance liées au code du commit *\\
    rapportBase & Architecture du rapport : feuille de style,
                  préambule, couverture, sous-dossiers, Makefile\\
    rapport & Autres parties du rapport
  \end{tabularx}

  \caption{Organisation des différents codes}
  \label{fig:git:hierarchie}
\end{figure*}

Puisqu'un commit enregistre l'état de tout un système de fichiers
(dossier et sous-dossiers), cet arborescence ne correspond pas à des
dossiers. Pour afficher le code de, par exemple, \texttt{mandel.c} en
charges dynamiques, il faut (dans n'importe quel ordre) :

\begin{itemize}
\item demander à Git d'afficher la branche \emph{mpi-dynamic} :\\
  \texttt{git checkout mpi-dynamic} ;\\ ainsi, le système de fichiers
  est identique à l'état sauvegardé dans le dernier commit de la
  branche \emph{mpi-dynamic} ;
\item se déplacer dans le sous-dossier \texttt{Mandelbrot} ; vous
  trouverez alors le \texttt{mandel.c} en équilibrage dynamique.
\end{itemize}

L'organisation du système de fichiers employée est décrite en figure
\ref{fig:git:syst}.

Un dépôt Git aussi structuré permet de factoriser les modifications du
code, donc permet une meilleure automatisation (voir l'annexe
\ref{app:Makefiles} qui utilise une telle automatisation). En
l'ocurrence, si je veux renommer la variable qui contient le rang du
processus, je me déplace sur \emph{mpi-master} (puisque cette
modification impacte tout ce qui a trait à MPI), je fais ma
modification, j'amende le dernier commit\footnote{plutôt que d'en
  créer un nouveau, car pour ce projet je préfère un historique facile
  à lire qu'un historique complet}, puis je rebase toutes les branches
qui partent de celui-ci sur sa version amendée. À présent, tous les
codes ont le nouveau nom de variable. Je peux par exemple faire
\texttt{grep <nouveau nom> mandel.c} pour voir à quoi elle est
assignée sur n'importe quelle branche.

\clearpage



\begin{onecolumn}

\section{Système de fichiers}

\begin{longtable}{>{\ttfamily}m{0.36\textwidth} p{0.58\textwidth}}
  \caption{Système de fichiers\label{fig:git:syst}}\\
  \null\hspace{0.15\textwidth}chemin & (branche) Description\\
  \hline
  \endfirsthead
  \caption{(continued)}\\
  \null\hspace{0.15\textwidth}chemin & (branche) Description\\
  \hline
  \endhead
    /                          & Racine du projet\\
    /.gitattributes            & Configuration de la manière dont Git
                                 doit interpréter chaque fichier.\\
    /.gitignore                & Configuration de la liste des motifs
                                 de fichier dont Git ne doit pas
                                 s'occuper.\\
    /.git/                     & Dossier de stockage pour Git\\
                               & \\
    /Convolution/              & Code relatif au TP sur la convolution\\
    /C /convol.c               & Code source\\
    /C /hosts                  & (\emph{mpi-master}) Liste des
                                 machines sur lesquelles exécuter le
                                 code. Une adresse IP ou nom par
                                 ligne.\\
    /C /Makefile               & Automatise la compilation du code
                                 (voir annexe \ref{app:Makefiles})\\
    /C /propagate              & (\emph{mpi-master}) Script
                                 automatisant le déploiement et
                                 l'exécution sur les machines dans le
                                 fichier hosts. Voir annexe
                                 \ref{app:propagate}\\
    /C /rasterfile.h           & Quelques constantes sur le format des
                                 fichiers \texttt{.ras}\\
                               & \\
    /Mandelbrot/               & Code relatif au TP sur la fractale\\
    /M /mandel.c               & Code source\\
    /M /hosts                  & idem\\
    /M /Makefile               & idem\\
    /M /propagate              & idem\\
    /M /rasterfile.h           & idem\\
                               & \\
    /Donnees/                  & Fichiers nécessaires à l'exécution de
                                 l'un ou l'autre des programmes
                                 ci-dessus\\
    /D /Albert-Einstein.ras    & Image sur laquelle lancer la
                                 convolution.\\
                               & \\
    /init-mpi                  & (\emph{mpi-master}) Script
                                 automatisant la génération de clés
                                 openssh sur les machines du
                                 fichier Mandelbrot/hosts\\
                               &\\
    /Resultats/                & Dossier contenant les fichiers CSV
                                 des mesures.\\
    /S /filtres.txt            & Nom des filtres implémentés dans
                                 \texttt{convol.c}\\
    /S /convol-*br*.csv        & (\emph{*br*}) Mesures du code de
                                 \texttt{convol.c} sur la branche
                                 *br*\footnote{Le suffixe *br* est
                                   nécessaire pour automatiser la
                                   construction du rapport — voir
                                   annexe \ref{app:Makefiles}}\\
    /S /mandel-*br*.csv        & (\emph{*br*}) Mesures du code de
                                 \texttt{mandel.c} sur la branche
                                 *br*\\
                               & \\
    /Rapport/                  & (\emph{master}) Contient les fichiers
                                 sources du rapport\\
    /R /appendix/              & Contient les fichiers sources des
                                 annexes\\
    /R /a /installation.tex    & Annexe A\\
    /R /a /git.tex             & Annexe B\\
    /R /a /makefiles.tex       & Annexe C\\
    /R /a /propagate.tex       & Annexe D\\
    /R /(data/)                & Contient les fichiers de mesure et
                                 code sources extraits des autres
                                 branches du dépôt après un
                                 \texttt{make}. Voir annexe
                                 \ref{app:Makefiles}.\\
    /R /fragment/              & Diverses parties du rapport\\
    /R /f /Albert-Einstein.jpg & Image\,\,convertie\,\,%
                                 d'après
                                 \texttt{Donnee/Albert-Einstein.ras}\\
    /R /f /CC-BY.png           & Logo © Creative Commons\\
    /R /f /copyright.tex       & Source du paragraphe en page 2\\
    /R /f /logo\_ensg.png      & Logo © École nationale des sciences
                                 géographiques\\
    /R /f /mandel.c            & Code utilisé en arrière-plan sur la
                                 couverture\\
    /R /f /rapport-convol.tex  & Rapport du TP~2\\
    /R /f /rapport-openmp.tex  & Rapport du TP~3\\
    /R /f /titlepage.tex       & Code source de la page de
                                 couverture\\
    /R /Makefile               & Fichier permettant d'automatiser la
                                 construction du rapport ; voir
                                 l'annexe \ref{app:Makefiles}\\
    /R /mandelbrot/            & Contient le code source du TP~1,
                                 séparé par question (de 0 à 4)\\
    /R /(rapport.aux)          & Lors de la première compilation,
                                 LaTeX y stocke les références, qui
                                 sont mises à jour lors de la deuxième
                                 compilation.\\
    /R /(rapport.log)          & LaTeX y stocke des détails sur la
                                 dernière compilation.\\
    /R /(rapport.lo*)          & LaTeX y stocke les références de
                                 figure, table, algorithmes, et
                                 listings entre deux compilations.\\
    /R /(rapport.out)          & LaTeX y stocke les liens de la table
                                 des matières du PDF entre deux
                                 compilations.\\
    /R /(rapport.pdf)          & Sortie de compilation\\
    /R /rapport.tex            & Code source principal du rapport
                                 incluant tous les autres.\\
    /R /(rapport.toc)          & LaTeX y stocke les liens de la table
                                 des matières entre deux
                                 compilations.\\
\end{longtable}

\end{onecolumn}

\clearpage

\twocolumn

%%% Local Variables:
%%% mode: latex
%%% TeX-master: "../rapport"
%%% End:
