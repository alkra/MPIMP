%% Alban Kraus
%% © 2016 École nationale des sciences géographiques
%% 6-8 avenue Blaise Pascal - Champs-sur-Marne
%% 77455 MARNE-LA-VALLÉE CEDEX 2

\section{Questions liminaires}

\subsubsection{La fonction convolution}

\begin{quotation}
  Dans la fonction \texttt{con\-vo\-lu\-tion(\-)}, pourquoi doit-on
  préparer un tampon intermédiaire au lieu de faire le calcul
  directement sur l'image ?
\end{quotation}

Le filtre s'applique à plusieurs pixels de l'image \emph{originale}
(ou de la convolution $i-1$). Si on fait le calcul directement sur
l'image (convolution $i$), il y aura au moins un de ces pixels qui
aura été altéré par le calcul précédent.

On a alors besoin de deux tampons : l'un contient l'image originale,
et l'autre l'image convoluée. À la fin du calcul, on remplace l'image
originale par l'image convoluée, qui sera l'image originale de la
prochaine convolution.


\subsubsection{Parallélisation de l'algorithme}

\begin{quotation}
  Quelles sont les séquences parallélisables de l'algorithme ?
\end{quotation}

Le calcul de la convolution ne dépend que de l'image originale : il
peut donc \^etre parallélisé. Autrement dit, le calcul de chaque pixel
est indépendant.

L'application du filtre ne peut pas \^etre parallélisée en règle
générale.

De plus, la composition des convolutions ne peut pas \^etre
parallélisée, car chacune dépend de la précédente.



\subsubsection{Équilibrage de charge}

\begin{quotation}
  Sachant que la taille du noyau $k$ de convolution est $3 \times 3$
  pixels, quelle est la complexité théorique du calcul d'un pixel de
  $I * k$ ? Quel type d'équilibrage de charge doit-on prévoir entre
  les processeurs ?
\end{quotation}

Pour chaque pixel, l'application du filtre se fait en temps constant
(ne dépend que de la taille du filtre). On peut donc appliquer une
répartition statique de charge.


\subsubsection{Découpage}

\begin{quotation}
  Quel découpage (répartition des données entre processeurs) est
  naturel dans ce contexte ?
\end{quotation}

Afin de minimiser les communications entre processus, et aussi parce
qu'il est plus simple à mettre en place, nous allons réaliser un
découpage en \emph{lignes}.


\subsubsection{Problèmes aux bords}

\begin{quotation}
  Quel problème (au bord des blocs de l'image) survient lors de
  l'itération de l'opération de convolution ?
\end{quotation}

Lors du calcul d'un bloc de l'image, chaque processus aura besoin de
la dernière ligne du processus précédent et de la première du suivant
pour calculer sa première ligne et sa dernière.

Chaque processus doit ainsi communiquer sa première ligne au processus
précédent et sa dernière ligne au processus suivant.


\subsubsection{Pseudo-code}

\begin{quotation}
  Implémenter un algorithme parallèle avec des envois bloquants de
  message.
\end{quotation}

\begin{algorithm}[t]
  \caption{Lecture de l'image}
  \label{alg:convol:stat:file}
  \Donnees{\;
    \begin{tabular}{l @{ : } l}
      \Rank & rang du processus (connu)\\
    \end{tabular}
  }

  \Variables{\;
    \begin{tabularx}{\linewidth}{l @{ : } X}
      $params$ & tableau [2] contenant les tailles des images\\
    \end{tabularx}
  }

  \Res{\;
    \begin{tabular}{l @{ : } l}
      \R & fichier image\\
      \H & hauteur de l'image\\
      \W & largeur de l'image\\
    \end{tabular}
  }

  \Deb{%
    \Si{\Rank == \MASTER}{%
      Lecture du fichier\;
      Récupération de $params$\;
    }

    \tcp{Envoi des paramètres}
    \hangindent=\skiptext\hangafter=1
    \Bcast{$params$, 2, \Int, \MASTER, \CommWorld}\;
    \H = $params$[0]\;
    \W = $params$[1]\;
  }
\end{algorithm}

Les algorithmes \ref{alg:convol:stat:file} et
\ref{alg:convol:stat:mem} ne nécessitent que peu d'explications. Le
processus maître lit l'image et l'envoie à ses esclaves. La fonction
\texttt{MPI\_Scatter} se charge de répartir statiquement la charge de
travail entre les processus esclaves.

Néanmoins, comme établi à la question précédente, cet envoi ne
concerne que les lignes propres au processus en question. Les échanges
de lignes se déroulent dans l'algorithme \ref{alg:convol:stat:calc}.

\begin{algorithm*}
  \caption{Allocation de la mémoire}
  \label{alg:convol:stat:mem}

  \Donnees{\R, \H, \W, \Rank\;
    \begin{tabular}{l @{ : } l}
      \P & nombre de processus (connu)\\
    \end{tabular}
  }

  \Res{\;
    \begin{tabularx}{\linewidth}{l @{ : } X}
      \Hlocal & hauteur d'un bloc\\
      \Ima & pointeur vers le début de l'image\\
    \end{tabularx}
  }

  \Deb{
    \hangindent=\skiptext\hangafter=1
    \Hlocal = \H / \P \\
    + (\Rank > 0 ? 1 : 0)\\
    + (\Rank < \P-1 ? 1 : 0)\;

    \uSi{\Rank == \MASTER}{
      \Ima = \R.data;
    }
    \Sinon{
      \hangindent=\skiptext\hangafter=1
      \Ima = \malloc{\Hlocal * \W * \sizeof{unsigned char}}\;
      Test de l'allocation\;
    }

    \hangindent=\skiptext\hangafter=1
    \Scatter{
      \Ima, \mbox{\W * \H / \P, \Char},
      \mbox{\Ima + (\Rank > 0 ? \W : 0)},
      \mbox{\W * \H / \P}, \Char, \MASTER, \CommWorld
    }\;
  }
\end{algorithm*}

Chaque processus différent du premier commence par envoyer sa deuxième
ligne au précédent. Pour éviter un blocage perpétuel, le deuxième
\texttt{if} doit commencer par la réception de la dernière
ligne. Explications dans le tableau \ref{tab:convol:stat:bloquant}.

Une fois l'échange de lignes effectué, le travail peut
commencer. Entre chaque application du filtre, les processus échangent
à nouveau leurs lignes extrêmes, car elles ont été modifiées.

À la fin du calcul, tous les processus envoient leur résultat sans la
première ni la dernière ligne, et \texttt{MPI\_Gather} se charge de
refabriquer l'image finale.

\begin{table*}
  \centering
  \begin{tabularx}{\linewidth}{X X X X}
    Processus 0 & Processus 1 & Processus 2 & Processus 3\\
    \hline
    Recv(1) \hfill $\leftarrow$  & Send(0)   & Send(1)     & Send(2)\\
    Send(1) \hfill $\rightarrow$ & Recv(0)   & Send(1)     & Send(2)\\
    \hline
    (travail) & Recv(2) \hfill $\leftarrow$  & Send(1)   & Send(2)\\
              & Send(2) \hfill $\rightarrow$ & Recv(1)   & Send(2)\\
    \hline
              & (travail) & Recv(3) \hfill $\leftarrow$  & Send(2)\\
              &           & Send(3) \hfill $\rightarrow$ & Recv(3)\\
    \hline
              &           & (travail)             & (travail)\\
  \end{tabularx}
  \caption{Échange bloquant des lignes}
  \label{tab:convol:stat:bloquant}
\end{table*}

\begin{algorithm*}
  \caption{Calcul et envoi du résultat}
  \label{alg:convol:stat:calc}

  \Donnees{\Ima, \W, \H, \Hlocal, \Rank, \P\;
    \begin{tabular}{l @{ : } l}
      \Filtre & numéro du filtre (commande)\\
      \Niter & nombre d'itérations (commande)\\
    \end{tabular}
  }

  \Variables{\;
    \begin{tabular}{l @{ : } l}
      $status$ & état de la requête\\
      $i$ & itération courante\\
    \end{tabular}
  }

  \Deb{%
    \Pour{$i$ de 0 à \Niter} {
      \tcp{Communications avec le processus précédent}
      \Si{\Rank > 0}{
        \Send{\Ima + \W, \W, \Char, \Rank-1, \Exchange, \CommWorld}\;
        \Recv{\Ima, \W, \Char, \Rank-1, \Exchange, \CommWorld,
          \&$status$}\;
      }

      \tcp{Communications avec le processus suivant}
      \Si{\Rank < \P-1}{
        \Recv{%
          \mbox{\Ima + (\Hlocal-1) * \W},
          \W, \Char, \Rank+1, \Exchange, \CommWorld, \&$status$
        }\;
        \Send{%
          \mbox{\Ima + (\Hlocal-2) * \W},
          \W, \Char, \Rank+1, \Exchange, \CommWorld
        }\;
      }

      calcul(\Filtre, \Ima, \Hlocal, \W)\;
    }

    \Gather{%
      \mbox{\Ima+\W},
      \mbox{\W * \H / \P}, \Char, \Ima,
      \mbox{\W * \H / \P}, \Char, \MASTER, \CommWorld
    }\;

    \Si{\Rank == \MASTER}{%
      Enregistrement de \Ima\;
    }
  }
\end{algorithm*}

\subsubsection{Analyse des performances}

Les tests ont été effectués sur l'image Albert-Einstein.ras en figure
\ref{fig:convol:einstein}, dont la provenance est douteuse. Cette
image a pour dimensions $1200 \times 1000$.

\begin{figure}
  \centering
  \includegraphics[width=\linewidth]{\fragment/Albert-Einstein}
  \caption{L'image de test}
  \label{fig:convol:einstein}
\end{figure}

Dans la figure \ref{fig:convol:iter}, on s'intéresse aux performances
de calcul en fonction du nombre d'itérations. Le filtre 4 est beaucoup
plus long à appliquer que les autres ; il fait donc l'objet d'un
diagramme séparé.

\paragraph{Le filtre 4}
Toutes les courbes sont linéaires et s'intersectent en $(0, 0)$. On
peut donc en lire la pente aisément : avec six machines, le filtre 2
a une pente de 4 secondes par millier d'itérations, et le filtre 4 de
120 secondes par millier d'itérations. La pente du filtre 4 est donc
trente fois plus grande que celle des autres filtres. Le tri des
valeurs pour en extraire la médiane peut être une source de calculs
supplémentaires ; pour valider cette hypothèse, il serait nécessaire
d'examiner le code assembleur et compter les opérations, ce qui me
paraît trop compliqué pour le cadre de ce TP.

\begin{figure}
  \centering

  \begin{tikzpicture}
    \begin{axis}[
      xlabel={Nombre d'itérations},
      ylabel={Temps de calcul (s)},
      legend style={
        at={(0.5,1)},
        anchor=south
      }
      ]

      \addplot[red, dashed] table [x index=3, y index=4,
      col sep=semicolon, header=false]{%
        \data/convol-iter-1-4.csv%
      };

      \addplot[red] table [x index=3, y index=4,
      col sep=semicolon, header=false]{%
        \data/convol-iter-6-4.csv%
      };

      \legend{filtre 4 (1 machine), filtre 4 (6 machines)}
    \end{axis}
  \end{tikzpicture}

  \begin{tikzpicture}
    \begin{axis}[
      xlabel={Nombre d'itérations},
      ylabel={Temps de calcul (s)},
      legend style={
        at={(0.5,1)},
        anchor=south
      }
      ]
      \addplot[black, dashed] table [x index=3, y index=4,
      col sep=semicolon, header=false]{%
        \data/convol-iter-1-0.csv%
      };

      \addplot[blue, dashed] table [x index=3, y index=4,
      col sep=semicolon, header=false]{%
        \data/convol-iter-1-2.csv%
      };

      \addplot[black] table [x index=3, y index=4,
      col sep=semicolon, header=false]{%
        \data/convol-iter-6-0.csv%
      };

      \addplot[blue] table [x index=3, y index=4,
      col sep=semicolon, header=false]{%
        \data/convol-iter-6-2.csv%
      };

      \legend{filtre 0 (1 machine), filtre 2 (1 machine), filtre 0 (6
        machines), filtre 2 (6 machines)}
    \end{axis}
  \end{tikzpicture}

  \caption{Temps de calcul en fonction du nombre d'itérations ; sur 1 ou 6 machines}
  \label{fig:convol:iter}
\end{figure}

Toutes les courbes concordent : le temps de calcul est linéaire par
rapport au nombre d'itérations. En effet, l'application du filtre se
fait en temps constant. Les écarts observés lorsqu'une seule machine
est utilisée doivent être expliqués par le fait que j'utilisais cette
machine pour taper mon rapport : tous les processus n'étaient donc pas
entièrement consacrés au calcul. Cela dit, cette hypothèse est à
tempérer, puisque ces mesures ont été acquises avec deux processus sur
une machine disposant de huit cœurs...


\begin{figure}
  \centering

  \begin{tikzpicture}
    \begin{axis}[
      xlabel={Nombre de processus},
      ylabel={Temps de calcul (s)},
      legend style={
        at={(0.5,1)},
        anchor=south
      }
      ]
      \addplot[black, dashed] table [x index=1, y index=4,
      col sep=semicolon, header=false]{%
        \data/convol-proc-1-0.csv%
      };

      \addplot[red, dashed] table [x index=1, y index=4,
      col sep=semicolon, header=false]{%
        \data/convol-proc-1-4.csv%
      };

      \addplot[black] table [x index=1, y index=4,
      col sep=semicolon, header=false]{%
        \data/convol-proc-6-0.csv%
      };

      \addplot[red] table [x index=1, y index=4,
      col sep=semicolon, header=false]{%
        \data/convol-proc-6-4.csv%
      };

      \legend{filtre 0 (1 machine), filtre 4 (1 machine), filtre 0 (6
        machines), filtre 4 (6 machines)}
    \end{axis}
  \end{tikzpicture}

  \caption{Temps de calcul en fonction du nombre de processus ; sur 1
    ou 6 machines}
  \label{fig:convol:proc}
\end{figure}

\paragraph{Nombre de processus}
Les courbes de la figure \ref{fig:convol:proc} ont la même forme que
celles obtenues au TP précédent. L'interprétation en est identique.

%%% Local Variables:
%%% mode: latex
%%% TeX-master: "../rapport"
%%% End:
