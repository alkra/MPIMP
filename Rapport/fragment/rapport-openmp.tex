%% Alban Kraus
%% © 2016 École nationale des sciences géographiques
%% 6-8 avenue Blaise Pascal - Champs-sur-Marne
%% 77455 MARNE-LA-VALLÉE CEDEX 2

\lstset{%
  basicstyle=\footnotesize,
}%

\section{Questions}

\subsection{Nombre de cœurs disponibles}

\begin{quotation}
  Vérifiez à l'aide de la commande \texttt{cat /proc/cpuinfo} le
  nombre de cœurs disponible sur votre machine.
\end{quotation}

Cette commande affiche les statistiques de \emph{huit} cœurs de
calcul~; il faut noter que l'hyperthreading est activé, donc il n'y a
en réalité qu'un processeur à quatre cœurs.

\subsection[Premier programme]{1\textsuperscript{er} programme}

\begin{quotation}
  Utiliser un des programmes du cours pour tester votre
  environnement. N'oubliez pas de positionner la variable
  \texttt{OMP\_NUM\_THREADS} !
\end{quotation}

\subsection{Calcul de fractales}

\begin{quotation}
  Le fichier \texttt{mandel.c} contient le code source d'un programme
  qui calcule les valeurs de l'espace de Mandelbrot.

  Parallélisez ce programme avec OpenMP, puis exécutez-le avec des
  équilibrages de charges statiques et dynamiques et avec différentes
  valeurs de taille de bloc.
\end{quotation}

Pour paralléliser ce calcul, on utilise l'interface de compilation
d'OpenMP (directives \texttt{\#pragma omp} et compilation avec
\texttt{gcc -fopenmp}). En particulier, on ajoute une première ligne
et un bloc d'accolades pour délimiter la zone à paralléliser, et on
parallélise une boucle \texttt{for}.

L'algorithme séquentiel a dû être légèrement modifié pour faire en
sorte que les modifications des variables locales ne soient pas
écrasées par les processus. Une variable \emph{privée} sera propre à
chaque processus, et chaque processus peut avoir une valeur différente
pour cette variable. Le code est affiché au listing
\ref{lst:omp:mandel:seq}.

\lstinputlisting[float=*, language=C, firstline=269,
lastline=283, caption=Code de Mandelbrot modifié pour OpenMP,
label=lst:omp:mandel:seq]{\fragment/mp-seq-mandel.c}

On peut même combiner OpenMP et OpenMPI, de manière à paralléliser
chaque processus. Par manque de temps, aucune mesure n'a pu être
prise, ce qui est bien dommage : il faudrait mesurer la différence (ou
plutôt, selon ma conjecture, l'absence de différence) entre six
processus non threadés, un processus avec six threads, et six
processus avec six threads sur une seule machine ; puis la différence
entre six processus non threadés et six processus threadés sur six
machines à huit cœurs. Le code est présenté au listing
\ref{lst:omp:mandel:stat}.

\lstinputlisting[float=*, language=C, firstline=304,
  lastline=322,
  caption=Code statique de Mandelbrot modifié pour OpenMP,
  label=lst:omp:mandel:stat]{\fragment/mp-stat-mandel.c}

Dans ce code, les lignes à générer sont partagées entre les processus,
puis chaque processus les partage à nouveau entre ses fils d'exécution.



\subsection{Convolution}

\begin{quotation}
  Le fichier \texttt{convol.c} contient le code source d'un programme
  qui applique des opérateurs de convolution à des images.

  Parallélisez ce programme avec OpenMP, puis exécutez-le avec des
  équilibrages de charge statiques et dynamiques et avec différentes
  valeurs de taille de bloc.
\end{quotation}

Le code modifié est présenté au listing \ref{lst:omp:convol:seq}. De
même que dans le TP précédent, c'est la boucle qui applique le filtre
qui est parallélisée. Sans raison véritable, j'ai parallélisé aussi la
boucle réalisant les copies de mémoire-tampon.

\lstinputlisting[float=*, language=C, linerange={255-267, 274-278},
caption=Code de convolution modifié pour OpenMP,
label=lst:omp:convol:seq
]{\fragment/mp-seq-convol.c}

%%% Local Variables:
%%% mode: latex
%%% TeX-master: "../rapport"
%%% End:
